\chapter{Introduction}
The Marangoni effect is a liquid flow along a fluid-fluid interface as a result of either a chemical potential gradient or a temperature gradient tangential to the dividing surface.
2015 saw the $160^{th}$ anniversary of J. Thompson's first scientific investigation into this effect and yet whilst that period has seen a significant advance in our understanding of the phenomenom, a fully microscopic description evades us.
Despite this, Marangoni flows are becoming ever more important with technological applications ranging from oil extraction deep within the Earth's crust to microfluidics on board satellites operating within low-gravity environments.
This study will aim to simulate these flows as a result of a temperature gradient using a molecular dynamics simulation.

\section{Early Studies} 
the ``teardrops'' that appear around the edge of a glass of wine provide the basis of the first study by J. Thompson into the Marangoni effect in 1855\cite{Thompson1855}.
In this paper, Thompson describes how the preferential evaporation of alcohol from a thin layer of wine around the edge of the glass creates a more dilute liquid, which in turn generates a surface tension gradient.
This gradient acts pull the wine in the bulk of the glass up the liquid-air interface and thus generates the flows observed within the glass.
The phenomenom was later studied by Carlo Maragoni after whom it gets its name.

Since then, a number of studies have been made covering a wide range of manifestations of the effect from the formation of rings after the evaporation of coffee drops to the lateral transportation of microdroplets.
Despite this large amount of research, the 

\section{Micro-gravity environments}
In low gravity environments, thermocapillary becomes an incredibly important mechanism for fluid transport.
On Earth, most fluid flow is driven by gravitational forces whilst in space, the low gravitational field causes surface driven effects to dominate.
These surface forces may arise from capillary effects as well as thermocapillary forces and they ma
Such flows are probably most commonly observed by the sharp-eyed wine connoisseur as teardrops forming along the edges of a wine glass just above the liquid surface, and this was the basis of the first study into the effect  

