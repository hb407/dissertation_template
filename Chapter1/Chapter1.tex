\section{Introduction}
2015 saw the $160^{\mathrm{th}}$ anniversary of J. Thompson's report on interfacial flows as the result of a chemical potential or temperature gradient.\cite{JThompson}
This phenomenom was later termed the ``Marangoni Effet'' after it formed the basis of Carlo Marangoni's doctoral dissertation.\cite{Marangoni}
Since then, temperature induced Marangoni flows, also known as thermocapillary flows, have been the subject of a number of studies and a macroscopic description has been developed through the works of Derjaguin\cite{SurfaceForces} and Levich\cite{Levich}, as summarised in the context of phoresis by Anderson\cite{Anderson}.

On the simplest level, this effect can be described as a consequence of a interfacial--tension gradient, with motion occuring from regions of low to regions of high surface--tenson.
However, this is not particulaly informative since the temperature dependence of surface--tension is itself known only through empirical studies.
Beyond this description, Derjaguin demonstrated that the flow can be calculated from the excess enthalpy of a fluid close to the interface, hence formulating a thermodynamic theory of thermocapillary motion.\cite{SurfaceForces}

The thermodynamic backbone upon which this theory is founded is reliant on knowledge of the macroscopic properties of the system.
In contrast, the flow itself is inherently a microscopic phenomenon and is localised at the interfaces of bulk fluids. 
Such a localised effect cannot be faithfully described using a thermodynamic description, instead it requires a microscopic theory formulated around the interpaticle interactions. 
No such theory currently exists, and there has been limited research in this area.\cite{HolgerBoppHampe}
By studying the microscopic properties of a liqud--liquid interface, this report aims to model the Maragnoni effect and investigate the link between the microscopic fluid properties and its macroscopic motion.

\subsection{Experimental studies}
Despite the lack of a microscopic theory, there has been substantial experimental research into the Marangoni effect.
Many of these studies have focussed on the more curious examples of Marangoni flows, such as Thompson's ``tears of wine''\cite{JThompson,Venerus,Tadmor,Cazabat1995} and the ``coffee ring effect'',\cite{Sefian,HuLarson,Sefiane2014} whilst others have shown it is becoming ever more important in technological applications.
For example, Sternling and Scriven\cite{SternlingScriven} proposed Marangoni effects as the origin for interfacial turbelence and mass transport, yielding applications in fluid mixing and oil recovery,\cite{Aguilera2005,LyfordA,LyfordB} whilst Subramanian and Balasubramanian outline the importance of Marangoni forces for the motion of bubbles and droplets in reduced gravity environments.\cite{MotionOfBubblesAndDrops} 

In 1959, Young et al. produced a theoretical description of this motion of bubbles and droplets under the influence of a temperature gradient, a phenomena described as thermophoresis.\cite{Young1959}
They describe how this gradient causes a higher surface--tension on the low temperature side of the droplet, resulting in a force pulling the surrounding fluid towards the low temperature region and a corresponding reaction force propelling the droplet towards the warmer fluid region (note the similarity to electrophoresis where there is no-net force within the droplet).
This force was measured experimentally by S. C. Hardy,\cite{Hardy1978} who used a temperature gradient to balance the Marangoni and buoyancy forces acting on a droplet within a fluid, thus holding the droplet stationary.
Later, theoretical modelling by Kim and Subramanian suggested that the inclusion of surface--active substances could be used to prevent thermophoresis,\cite{KimSubramanianA,KimSubramanianB} a prediction that was confirmed experimentally.\cite{BartonSubramanian,ChenStebe}

With the advances in space technology over the past half a century, thermophoresis and thermocapillary motion have become important mechanisms for fluid motion and mass transfer in low--gravity environments, where surface effects dominate over buoyancy driven motion.
The motion of bubbles and drops due to interfacial gradients is considered to be key for materials processing in space, enabling phase separation of binary mixtures and the potential to make uniform composite materials.\cite{BartonSubramanian}
Moreover, fluid transport in the absence of gravity is important for controlling fluid fuels aboard satellites, often carried for thrusters used to perform orbital adjustments.\cite{MotionOfBubblesAndDrops} 

\subsection{Striving for a microscopic description}
With so many potential applications, the search for a micrcoscpic description is becoming ever more significant, and computer simulations are increasingly being used to understand this phenomenom.
Marangoni flows are inherently dynamic and must be studied using a time--dependent method such as molecular dynamics.
To model the dynamic behaviour of an immiscible binary--mixture under the effect of a temperature gradient might initially appear trivial; simply create a temperature gradient for such a system and measure the subsequent particle velocites.
This non-equilibirum approach is used by Hampe et al. in their study of the Marangoni effect.\cite{HolgerBoppHampe}
Despite their positive results, the method is complicated by the use of periodic boundary conditions (required to model a macroscopic fluid) that require each unit cell to have two temperature gradients, such that the temperature is also periodic.
Any flow due to this gradient then generates an opposing concentration gradient and an equilibrium stationary state will be reached.

In this study we investigate the use of equilibirium molecular dynamics simulations for modelling the Marangoni effect.
%Molecular dynamics heavily relies on the use of forces acting on a particle, these forces are calculated on each timestep and are used to integrate the equations of motion.
%It is therefore easy to measure the stress acting on a fluid element during a simulation or to apply artificial forces.
By calculating the equilibrium stress acting on a binary mixture for two different temperatures, the differential of the stress with respect to temperature can be estimated and used to infer a body force corresponding to the Marangoni force.
This can then be applied as a constant body force in a non-equilibrium simulation at an intermediate temperature to imitate the Marangoni force, thus circumventing the complications of a periodic temperature profile.
Finally, by measuring the velocity profile in this non--equilibrium simulation, the Marangoni flow can be observed.

Using this equilibrium method we show that a Marangoni force at liquid--liquid interface can be calculated from the fluid stress--tensor and used to generate a Marangoni flow.
This technique is then used to investigate the effect of surfactants on the magnitude of the Marangoni flow and compare these results to the experimental observations.

The theoretical description of the Maragoni effect is described in Section 2, whilst Section 3 outlines the computational methods used.
The simulation results are discussed in Section 4.
Section 5 provides a summary of the conclusions and proposes directions for future studies.
