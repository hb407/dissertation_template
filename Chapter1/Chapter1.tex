\section{Introduction}
2016 saw the $161^{\mathrm{st}}$ anniversary of J. Thompson's first scientific report on the observation of flows occuring at a fluid--fluid interface as a result of a tangential chemical potential or temperature gradient, a phenomenom later termed the ``Marangoni Effet'' after it formed the basis of Carlo Marangoni's doctoral dissertation.
Since then, temperature induced Marangoni flows have been the subject of a number of studies and a comprehensive macroscopic description has been developed through the works of Derjaguin\cite{SurfaceForces} and Levich\cite{Levich}, as summarised by Anderson\cite{Anderson}.
On the simplest level this effect can be described as a flow occuring due to a gradient in the surface--tension at the interface, although this is not particulaly informative since the temperature dependence of surface--tension is itself known only through empirical studies.
Beyond this, Derjaguin demonstrates that the flow can be calculated from the excess enthalpy of a fluid close to the interface.

In all of these studies, the thermodynamic backbone upon which the theories within are founded is reliant on a knowledge of the macroscopic properties of the system, and yet the Marangoni effect inherently a microscopic phenomena localised at the interface.
Such a localised effect cannot be truthfully described using such macroscpic descriptions, instead a microscopic description is required. 
There is currently no full microscopic description of the Marangoni effect and limited studies focussing on this area.

Despite this, there has been substantial experimental research into the Marangoni effect and it is becoming ever more important in technology. 
In particular, Marangoni forces can provide thermocapillary motion which yields an important mechanism for fluid transport in low--gravity envrironments where surface driven effects dominate.
APPLICATIONS


This study focusses on modelling Marangoni flows using a microscopic method to gain a better understanding of the relationship between the flow properties and the microscopic nature of the system.
The Marangoni effect is inherently a dynamic phenomenom and must be studied using a time--dependent method.
Molecular dynamics (MD) simulations model a system classically by solving Newton's equations of motions, allowing temporal averages of the system properties to be calculated.\cite{Bopp2008}
The ergodic hypothesis suggests that equilibrium time--averages correspond to ensemble averages for a sufficiently long simulation time.

To model the dynamic behaviour of a binary--mixture under the effect of a temperature gradient might initially appear trivial; simply create a temperature gradient for such a system and measure the expected flow velocity of the resulting particles.
This non-equilibirum approach is used by Hampe et al. in their study of the Marangoni effect.\cite{HolgerBoppHampe}
Despite their positive results, this method is significantly complicated by the use of periodic boundary conditions (required to model a macroscopic fluid).
As a result of these boundary conditions, each unit cell must in fact have two temperature gradients such that the temperature is also periodic.
Any flow due to this gradient will then rapidly generate an opposing concentration gradient and an equilibrium will be reached.

Instead it is also possible to model the Marangoni effect using the results of equilibrium simulations. 
Molecular dynamics heavily relies on the use of forces acting on a particle; these forces are calculated on each timestep and used to integrate the equations of motion.
As a result, it is easy to measure the stress acting on a fluid element during a simulation and also to apply any required artificial forces.
By calculating the stress acting on a binary mixture at equilibrium for two different temperatures, the differential of the stress with respect to temperature can be estimated and used to infer a body force corresponding to the Marangoni force.
This force can then be applied as a constant body force in a non-equilibrium simulation on a fluid at an intermediate temperature to imitate the Marangoni force, thus circumventing the complications of a periodic temperature profile.
Finally, by measuring the velocity profile of this non--equilibrium simulation the Marangoni flow can be observed.
It is this equilibrium method that is employed in this study of the Marangoni effect.
