\section{Introduction}
Interfacial flows resulting from a chemical potential or temperature gradient were first reported by J. Thompson in 1855.\cite{JThompson}
After forming the basis of Carlo Marangoni's doctoral dissertation in 1865, this phenomenon was termed the ``Marangoni Effect''.\cite{Marangoni}
Subsequently, many scientific studies have focussed on temperature induced Marangoni flows, also known as thermocapillary motion

Through the works of Derjaguin et al.\cite{SurfaceForces} and Levich\cite{Levich}, a thorough macroscopic description has been developed.
Anderson summarises these results in the context of phoresis.\cite{Anderson}

The Marangoni effect can be described as resulting from an interfacial tension gradient.
Experimental studies indicate hot fluids have a lower surface tension than cold fluids,\cite{Ficalbi1972,Kayser1975} inducing motion in the opposite direction to a temperature gradient.
However, this analysis does not progress far since the temperature dependence of surface tension is not rigorously understood.

Levich used hydrodynamics to derive the fluid velocity across a shallow pan in terms of $\partial \gamma / \partial T$.\cite{Levich}
He achieved this using a continuity equation setting the total flow to zero.
This implied that any interfacial flows are accompanied by an opposing flow in the bulk fluid.

In contrast, Derjaguin et al. calculated the momentum flux from a temperature gradient by computing the energy flux carried by pressure--driven convection and applying Onsager's reciprocal rule.\cite{SurfaceForces}
They showed that the velocity can be related to the excess enthalpy density near the interface (relative to the bulk), yielding a macroscopic description.

However, these theories are founded upon a thermodynamic backbone which relies on knowledge of the macroscopic system.
Marangoni flows are a microscopic phenomenon, localised at fluid interfaces, and cannot be faithfully described using thermodynamics.
Instead, a microscopic theory using interparticle interactions is required.
No such theory currently exists, and there has been limited research in this area.\cite{HolgerBoppHampe}

By studying the microscopic properties of a liquid--liquid interface, this report aims to model the Maragnoni effect and investigate the link between microscopic fluid properties and its macroscopic motion.

\subsection{Experimental studies}
Despite the lack of a microscopic theory, there has been substantial experimental research into the Marangoni effect.
Many studies have focussed on the more curious examples of Marangoni flows, including Thompson's ``tears of wine''\cite{JThompson,Venerus,Tadmor,Cazabat1995} and the ``coffee ring effect'',\cite{Sefian,HuLarson,Sefiane2014} whilst others demonstrate its importance in technology.
For example, Sternling and Scriven\cite{SternlingScriven} proposed Marangoni effects as the origin of interfacial turbulence and mass transport, yielding applications in fluid mixing and oil recovery.\cite{Aguilera2005,LyfordA,LyfordB} 
Furthermore, Subramanian and Balasubramanian outlined the importance of Marangoni forces for the motion of bubbles and droplets in reduced gravity.\cite{MotionOfBubblesAndDrops} 

In 1959, Young et al. produced a theoretical description of this motion of bubbles and droplets due to a temperature gradient, a phenomenon termed thermophoresis.\cite{Young1959}
They described how this gradient causes a higher surface tension on the low temperature side of the droplet, creating a force pulling the surrounding fluid towards this region.
A corresponding reaction force then propels the droplet towards the warmer fluid.
Analogously to electrophoresis, no net force acts on the fluid within the droplet.

This was measured experimentally by S. C. Hardy,\cite{Hardy1978} who used a temperature gradient to balance the Marangoni and buoyancy forces acting on a droplet within a fluid, thus holding the droplet stationary.
Later theoretical modelling by Kim and Subramanian suggested that the inclusion of surfactants could prevent thermophoresis,\cite{KimSubramanianA,KimSubramanianB} a prediction subsequently confirmed experimentally.\cite{BartonSubramanian,ChenStebe}

With the advances in space technology, thermophoresis and thermocapillary motion have become important mechanisms for fluid motion and mass transfer in low--gravity environments, where surface effects dominate over buoyancy driven motion.
The motion of bubbles and drops due to interfacial gradients is considered key for materials processing in space, enabling phase separation of mixtures and the potential to make uniform composite materials.\cite{BartonSubramanian}
Moreover, fluid transport in the absence of gravity is important for controlling fluids aboard satellites. 

\subsection{Striving for a microscopic description}
With many potential applications, the search for a microscopic description is becoming ever more significant, and computer simulations are increasingly being used to understand this phenomenon.
Marangoni flows are dynamic and can be studied using a time--dependent method such as molecular dynamics.
Modelling the behaviour of a partially miscible binary--mixture under the effect of a temperature gradient might appear trivial; create a temperature gradient in a system and measure the subsequent particle velocities.
Hampe et al. used this non-equilibrium approach in their study of the Marangoni effect.\cite{HolgerBoppHampe}
Despite positive results, this method is complicated by periodic boundary conditions that require the simulation box to have two temperature gradients, creating a periodic temperature gradient.
Any induced flow then generates an opposing concentration gradient, and an equilibrium state is reached.

In the present study, I investigate the use of equilibrium molecular dynamics simulations for modelling the Marangoni effect.
By calculating the equilibrium stress acting on a binary--mixture for two different temperatures, the temperature derivative of the transverse stress can be estimated and used to compute a body force.
This can then be applied in a simulation at an intermediate temperature to imitate the Marangoni force, thus circumventing the complications of a periodic temperature profile.
By measuring the velocity profile in this final simulation, Marangoni flows can be observed.

Using this equilibrium method, I show that a Marangoni force at a liquid--liquid interface can be calculated from the transverse stress tensor and used to generate a Marangoni flow.
The effect of surfactants on the magnitude of this flow is then investigated.

A theoretical description of the Marangoni effect is described in Section 2, whilst Section 3 outlines the computational methods used.
The simulation results are discussed in Section 4.
Section 5 summarises the conclusions and proposes directions for future studies.
