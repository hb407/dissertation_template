\chapter{Theoretical Background}
\section{A macroscopic description}
The motion of fluids may be described macroscopically using the  Navier--Stokes equation,\cite{simpleLiquids} modelling a liquid as a continuous medium and combining the conservation of mass and momentum with the relation between the force on a volume element and the local fluid flow.
In the low Reynolds-number regime the flow velocity is small and thus for a system at equilibrium, the Navier--Stokes equation reduces to
\begin{equation}
\label{NavierStokes}
\eta \nabla^{2}\mathbf{\nu}(r,t) = \nabla P(r,t) - \mathbf{f}(r).
\end{equation}
Crucially Eqn. \ref{SimpleLiquids} shows that motion in fluids \textit{can only} occur under the presence of a pressure gradient or external forces.
In the case of Marangoni flows, motion due to the temperature gradient can only occur via an induced local pressure gradienti, which may be derived from thermodynamics.

Consider a binary mixture with an interface at $z=0$.
The Gibbs--Duhem equation gives
\begin{equation}
V dP = \sum_{i=0}^{n} N_{i} d\mu_{i} + S dT.
\end{equation}
In the bulk, the fluid pressure is constant and isotropic, giving
\begin{equation}
\label{GibbsDuhem}
\left( \frac{\partial P}{\partial x}\right) = \sum_{i=0}^{n} \rho_{i}^{B} \left(\frac{\partial \mu_{i}}{\partial x}\right) + \frac{S}{V} \left( \frac{\partial T}{\partial x}\right).
\end{equation}
The Maxwell equations allow the entropy to be related to the partial differential of chemical potential with respect to temperature through
\begin{equation}
\left( \frac{\partial\mu_{i}}{\partial T} \right)_{P,N_{i}} = - \left(\frac{\partial S}{\partial N_{i}}\right)_{P,T}.
\end{equation}
Now the total entropy is the weighted sum of the partial entropy of the species in the system, 
\begin{equation}
S = \sum_{i=0}^{n}s_{i}N_{i},
\end{equation}
thus
\begin{equation}
\left( \frac{\partial\mu_{i}}{\partial T} \right)_{P,N_{i}} = - s_{i}.
\end{equation}
\\
Overall, this allows us to rewrite Eqn. \ref{GibbsDuhem} as
\begin{equation}
\left( - \sum_{i=1}^{n}\rho_{i}^{B}s_{i}^{B}+\frac{S^{B}}{V}\right)\left(\frac{\partial T}{\partial x}\right)
\end{equation}
for which the solution is trivially 
\begin{equation}
\frac{S^{B}}{V} = \sum_{i=1}^{n}\rho_{i}^{B}s_{i}^{B}.
\end{equation}
\\
Consider the case close to an interface or surface where the pressure will deviate from its bulk value.
Assume that $\mu_{i}$ and $T$ such that
\begin{equation}
\left(\frac{\partial \mu_{i}}{\partial T}\right) = - s_{i}^{B}
\end{equation}
remains true, then the pressure gradient reduces to
\begin{equation}
\left(\frac{\partial P(z,x)}{\partial x}\right) = \left(\sum_{i=1}^{n}\rho_{i}(z)\left(s_{i}(z)-s_{i}^{B}\right)\right)\left(\frac{\partial T}{\partial x}\right).
\end{equation}

Now in the case of an ideal mixture the partial entropy is given by $s_{i}=\frac{\mu_{i}-h_{i}}{T}$, and since $\mu_{i}$ and $T$ are independent of $z$, the pressure gradient can be expressed in terms of the excess enthalpy, $\Delta h(z)$.
\begin{equation}
\label{ExcessEnthalpy}
\left(\frac{\partial P(z,x)}{\partial x}\right)= - \Delta h(z)\frac{1}{T} \left( \frac{\partial T}{\partial x} \right) 
= - \Delta h(z) \left( \frac{\partial \ln T}{\partial x} \right).
\end{equation}

Substituting Eqn. \ref{ExcessEnthalpy} into Eqn. \ref{NavierStokes} and assuming there is no pressure gradient in either the $z$ or $y$ direction allows the velocity at the interface to be computed as
\begin{equation}
\label{DoubleIntegral}
\nu (z=0) = - \frac{1}{\eta}\int_{0}^{\infty} dz z \Delta h(z) \frac{\partial T}{\partial x}
\end{equation}
where it has been assumed that the fluid in the bulk is at rest ($\nu_{x}(z)=\infty$) is at rest.
This expression is almost equivalent to that derived by Derjaguin et. al. except for a factor of two, suggesting that Derjaguin's expression may be wrong.\cite{SurfaceForces, Anderson}

\section{A microscopic approach}
By its very nature the Marangoni effect is a phenomenom which occurs on microscopic lengthscales, and hence the macroscopic hyrdodynamical approach culminating in Eqn. \ref{DoubleIntegral} is not entirely appropriate.
Instead of solving the Navier--Stokes equation, the velocity flow field should be  related directly to the local forces acting on the fluid, which may be calculated from the local stress tensor,
\begin{equation}
\label{ForceStress}
f_{x}(z) = - \left( \frac{\partial \sigma_{xx}(z,x)}{\partial x} \right).
\end{equation}

In the case of Marangoni flows, the force is due to a temperature gradient and $f_{x}(z)$ can be calculated from the chain rule as
\begin{equation}
\label{ForceStressTemp}
f_{x}(z) = - \left( \frac{\partial \sigma_{xx}(z,x)}{\partial T} \right) \left( \frac{\partial T}{\partial x} \right).
\end{equation}
Eqn. \ref{ForceStressTemp} allows the body force to be inferred from the temperature gradient of the stress tensor.
With this it should in principle be possible to compute a flow profile as follows:
\begin{enumerate}
	\item Compute $\sigma_{xx}(z)$ for an equilibrium system at a given temperature $T_{1}$.
	\item Repeat the calculation for another equilibrium system at slightly higher temperate $T_{2}$.
	\item Approcimate the temperature gradient of the stress tensor using a finite difference approach,
		$$\left( \frac{\partial \sigma_{xx}(z,x)}{\partial T} \right) \approx \frac{\sigma_{xx}^{T_{2}}(z,x) - \sigma_{xx}^{T_{1}}(z,x)}{T_{2} - T_{1}}.$$
	      In this step it is important that the temperature difference is small such that the finite difference approach provides a good estimate of the stress gradient.
	\item Infer $f_{x}(z)$ using Eqn. \ref{ForceStressTemp}
	\item Compute flow profile by applying $f_{x}(z)$ as an artifical body force to a non-equilibrium simulation at a suitable intermediate temperature $T_{3}$ (i.e. $T_{1} < T_{3} < T_{3}$).
\end{enumerate}

It is this approach that is employed in the subsequent investigation of the Marangoni effect. 

\section{Non-uniqueness of the pressure tensor}



