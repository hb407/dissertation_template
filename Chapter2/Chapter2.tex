\chapter{Theoretical Background}
\section{A macroscopic description}
The motion of fluids may be described macroscopically using the  Navier--Stokes equation,\cite{simpleLiquids} modelling a liquid as a continuous medium and combining the conservation of mass and momentum with the relation between the force on a volume element and the local fluid flow.
In the low Reynolds-number regime the flow velocity is small and thus for a system at equilibrium, the Navier--Stokes equation reduces to
\begin{equation}
\label{NavierStokes}
\eta \nabla^{2}\mathbf{\nu}(r,t) = \nabla P(r,t) - \mathbf{f}(r).
\end{equation}
Crucially Eqn. \ref{NavierStokes} shows that motion in fluids \textit{can only} occur under the presence of a pressure gradient or external forces.
In the case of Marangoni flows, motion due to the temperature gradient can only occur via an induced local pressure gradienti, which may be derived from thermodynamics.

Consider a binary mixture with an interface at $z=0$.
The Gibbs--Duhem equation gives
\begin{equation}
V dP = \sum_{i=0}^{n} N_{i} d\mu_{i} + S dT.
\end{equation}
In the bulk, the fluid pressure is constant and isotropic, giving
\begin{equation}
\label{GibbsDuhem}
\bigg(\frac{\partial P}{\partial x}\bigg) = \sum_{i=0}^{n} \rho_{i}^{B} \bigg(\frac{\partial \mu_{i}}{\partial x}\bigg) + \frac{S}{V} \bigg(\frac{\partial T}{\partial x}\bigg).
\end{equation}
The Maxwell equations allow the entropy to be related to the partial differential of chemical potential with respect to temperature through
\begin{equation}
\bigg( \frac{\partial\mu_{i}}{\partial T} \bigg)_{P,N_{i}} = - \bigg(\frac{\partial S}{\partial N_{i}}\bigg)_{P,T}.
\end{equation}
Now the total entropy is the weighted sum of the partial entropy of the species in the system, 
\begin{equation}
S = \sum_{i=0}^{n}s_{i}N_{i},
\end{equation}
thus
\begin{equation}
\bigg( \frac{\partial\mu_{i}}{\partial T} \bigg)_{P,N_{i}} = - s_{i}.
\end{equation}
\\
Overall, this allows us to rewrite Eqn. \ref{GibbsDuhem} as
\begin{equation}
\bigg( - \sum_{i=1}^{n}\rho_{i}^{B}s_{i}^{B}+\frac{S^{B}}{V}\bigg)\bigg(\frac{\partial T}{\partial x}\bigg)
\end{equation}
for which the solution is trivially 
\begin{equation}
\frac{S^{B}}{V} = \sum_{i=1}^{n}\rho_{i}^{B}s_{i}^{B}.
\end{equation}
\\
Consider the case close to an interface or surface where the pressure will deviate from its bulk value.
Assume that $\mu_{i}$ and $T$ such that
\begin{equation}
\bigg(\frac{\partial \mu_{i}}{\partial T}\bigg) = - s_{i}^{B}
\end{equation}
remains true, then the pressure gradient reduces to
\begin{equation}
\bigg(\frac{\partial P(z,x)}{\partial x}\bigg) = \Bigg(\sum_{i=1}^{n}\rho_{i}(z)\Big(s_{i}(z)-s_{i}^{B}\Big)\Bigg)\bigg(\frac{\partial T}{\partial x}\bigg).
\end{equation}

Now in the case of an ideal mixture the partial entropy is given by $s_{i}=\frac{\mu_{i}-h_{i}}{T}$, and since $\mu_{i}$ and $T$ are independent of $z$, the pressure gradient can be expressed in terms of the excess enthalpy ($\Delta h(z)$).
\begin{equation}
\label{ExcessEnthalpy}
\bigg(\frac{\partial P(z,x)}{\partial x}\bigg)= - \Delta h(z)\frac{1}{T} \bigg( \frac{\partial T}{\partial x} \bigg) 
= - \Delta h(z) \bigg( \frac{\partial \ln T}{\partial x} \bigg).
\end{equation}

Substituting Eqn. \label{ExcessEnthalpy} into Eqn. \ref{NavierStokes} and assuming there is no pressure gradient in either the $z$ or $y$ direction allows the velocity at the interface to be computed as
\begin{equation}
\label{DoubleIntegral}
\nu (z=0) = - \frac{1}{\eta}\int_{0}^{\infty} dz z \Delta h(z) \frac{\partial T}{\partial x}
\end{equation}
where it has been assumed that the fluid in the bulk is at rest ($\nu_{x}(z)=\infty$) is at rest.
This expression is almost equivalent to that derived by Derjaguin et. al. except for a factor of two, suggesting that Derjaguin's expression may well be wrong\cite{SurfaceForces, Anderson}.

\section{The microscopic picture}

\section{Virial versus Kirkwood pressure}



