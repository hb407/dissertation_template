\section{Concluding Remarks}
Using equilibrium simulations at two close temperatures, the temperature derivative of the tangential stress was estimated for binary--mixtures through a finite difference method.
The derivatives of the virial and Irving--Kirkwood stresses were calculated for a mixture confined between two walls, and in both cases there was a similar negative interfacial peak.
Combining with a temperature gradient, an artificial body force was computed and applied in a third equilibrium simulation, resulting in a non--equilibrium with a net interfacial flow.
This was interpreted as a Marangoni flow and occurred in the opposite direction to the applied temperature gradient, as expected.
The confining walls provided a momentum sink, exerting a frictional force on the fluid and allowing a steady state flow to be achieved.

The same method was then applied for a binary--mixture periodic in all dimensions, and a similar peak in the stress derivative was observed.
However, the fine structure of this derivative was obscured by significant noise.
This was reduced by running the simulation for a longer time period.
Consequently, only the virial stress was feasible for calculating a sufficiently precise gradient profile.

In the absence of a momentum sink, the stress derivative was adjusted to ensure no net force was applied on the system, and was used to calculate an artificial body force.
Applying this force produced a flow profile with a net centre--of--mass motion despite the adjustment applied to remove a net force.

The origin of the centre--of--mass motion is unclear. 
It could arise from errors induced by noise in the applied force.
For example, errors in the average derivative will cause the derivative profile to be incorrectly recentred, and a net applied force could result.
Running the initial equilibrium simulations for an even longer period could allow the effect of noise to be verified. 

Alternatively, the centre--of--mass could be artificially fixed throughout the simulation.
However, observing a stationary centre--of--mass is essential to verifying the correct behaviour of Maragnoni flows, and fixing the centre--of--mass would obscure this.
Fixing the centre--of--mass would allow only the relative motion of different fluid regions to be observed.

Having established an effective model using the fluid confined between two walls, the effect of surfactants was investigated by adding different concentrations of non-ionic surfactant molecules into the interfacial region.
The tangential virial stress was computed at two temperatures and its temperature derivative was estimated through the finite difference.
As the concentration of surfactant increased, the magnitude of the negative interfacial peak was reduced and two secondary positive peaks on either side of the interface emerged.

The stress derivatives were used to calculate an artificial body force that was applied in a third equilibrium simulation.
The magnitude of the resulting Marangoni flow decreased to zero as the surfactant concentration increased, in agreement with experimental studies.
Furthermore, the shape of the velocity profile deviated from the linear decay observed in the absence of surfactant.
This may be due to a non--uniform fluid viscosity, since the addition of surfactants should decrease the viscosity of the interfacial region.

\subsection{Future directions of study}
There are a number of ambiguities in the results reported which further study could verify. 
For example, more accurate force profiles from the Irving--Kirkwood stress could be computed using a larger system.
This could be made more efficient by implementing the Irving--Kirkwood analysis in parallel.

Furthermore, by introducing a contribution from the harmonic bonding into the Irving--Kirkwood analysis, it could be used to study systems with added surfactant.
This would confirm whether the secondary peaks observed as surfactant fraction increases are the result of an increase in local density due to the presence of surfactant particles.

It would also be interesting to compute the viscosity across the liquid and compare this for the pure fluid and the fluid with surfactant present.
As described in Section \ref{SurfResult}, this can be calculated using a Green--Kubo relation.
Comparing the shape of the flow profile to the uniformity in the viscosity would allow the deviation from the Couette flow model to be investigated.

Beyond this, linear response theory suggests that the flow induced by a temperature gradient could computed as:
\begin{equation}
\left< v_{x} \right> = \frac{\nabla T}{T} V \int_{\tau =0}^{\infty} \left< J_{x}(0) Q_{x}(\tau) \right> \mathrm{d} \tau,
\label{LinResponse}
\end{equation}
where $J$ and $Q$ are the mass and heat flux respectively.
By computing this average velocity for the mixture confined between two walls, both with and without an applied body force, Equation \ref{LinResponse} could allow a general relation between flow velocity and a temperature gradient to be developed.
This could be compared for the bulk and interfacial regions, since there should be no correlation between the heat flux and mass flux away from the interfaces.

Furthermore, a greater understanding could be developed using Onsager's reciprocal rule, as employed be Derjaguin et al.
This could be achieved by applying a pressure gradient to a simulation and measuring the resulting heat flux.
From the relation of these quantities, the mass flux from a temperature gradient could be computed. 

Finally, once informative methods for the basic study of Marangoni flows have been developed, the natural progression would be to investigate different systems and geometries.
For example, it would be interesting to study a non--symmetric binary--mixture or to replicate the thermophoresis of droplets under a temperature gradient, as observed experimentally.
\\
\\
\indent Through the results of this study, combined with future work, a greater understanding of the relationship between the stresses within a fluid and the velocity due to a temperature gradient can be achieved. 
Ultimately this is working towards a microscopic description and a more comprehensive understanding of the Marangoni effect.
